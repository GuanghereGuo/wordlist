%===========================================================
% 软件著作权说明书 - LaTeX源文件
% 编译器:XeLaTeX
%===========================================================

\documentclass[14pt]{mancls} % 调用 mancls 文档类 (不加 code 选项即为说明书)

\usepackage{graphicx} % 图片支持
\usepackage{float}    % 强制图片位置
\graphicspath{{figures/}} % 图片路径

%-----------------------------------------------------------
% 软件名称
%-----------------------------------------------------------
\title{基于Excel的英语词汇辅助学习系统V1.0}

\begin{document}
\zihao{-4}

% 标题页和目录页由模板自动生成

%===========================================================
% 正文开始
%===========================================================

\section{引言}

\subsection{编写目的}
本手册旨在为“基于Excel的英语词汇辅助学习系统V1.0”的用户提供详细的安装、配置及操作指南。本系统专为英语学习者、教育工作者及语言研究人员设计,旨在解决传统单词学习中人工查询释义繁琐、数据整理效率低下、复习资料格式混乱等痛点。通过阅读本手册,用户可以掌握软件的全流程操作,实现个性化词汇库的自动化构建。

\subsection{软件概述}
本系统是一款集成了数据采集、自然语言处理与文档自动化管理的桌面端软件。系统采用 Python 语言开发,深度整合了 Excel 文档处理技术与网络爬虫技术。其核心功能是读取用户指定的 Excel 单词表,自动连接在线词典接口,精准抓取单词的音标、释义及权威例句,并将数据智能回填至原表格中。

此外,系统还具备以下企业级特性:
\begin{itemize}
    \item \textbf{数据安全机制}:在每次写入操作前自动生成带时间戳的备份文件,彻底杜绝数据丢失风险。
    \item \textbf{断点续传能力}:自动识别已处理状态,网络中断或程序崩溃后重启即可接着上次进度继续运行。
    \item \textbf{无损格式化}:在美化排版的同时,智能保留用户原有的单元格颜色标记,尊重用户的学习笔记习惯。
\end{itemize}

\section{运行环境与安装}

\subsection{硬件要求}
本软件对硬件配置要求较低,普通办公电脑即可流畅运行:
\begin{itemize}
    \item \textbf{处理器}:Intel Core i3 或同级别 AMD 处理器及以上。
    \item \textbf{内存}:建议 4GB RAM 及以上。
    \item \textbf{硬盘空间}:软件主体占用约 50MB,建议预留 500MB 以上空间用于存储备份文件。
    \item \textbf{网络连接}:必须连接互联网(推荐宽带或稳定的 Wi-Fi),用于实时获取词典数据。
\end{itemize}

\subsection{软件环境}
\begin{itemize}
    \item \textbf{操作系统}:Windows 10 / Windows 11 (64位)。
    \item \textbf{运行依赖}:Python 3.8 及以上版本。
    \item \textbf{必要运行库}:OpenPyXL, Requests, BeautifulSoup4, PyYAML。
    \item \textbf{办公软件}:Microsoft Excel 2016+ 或 WPS Office(用于查看结果)。
\end{itemize}

\subsection{安装与目录结构}
本软件为绿色免安装版,用户只需将软件包解压至任意无中文路径的文件夹即可。标准的目录结构如下所示:

%-----------------------------------------------------------
% 图片位置 1:文件结构图
% 请截取包含 src, xlsx, config 文件夹的文件资源管理器界面
%-----------------------------------------------------------
\begin{figure}[H]
    \centering
    \includegraphics[width=0.8\textwidth]{file_structure.png}
    \caption{软件安装目录结构示意图}
    \label{fig:structure}
\end{figure}

如图 \ref{fig:structure} 所示:
\begin{itemize}
    \item \textbf{config/}:存放系统配置文件 \texttt{config.yaml}。
    \item \textbf{xlsx/}:存放用户的数据源文件 \texttt{words.xlsx} 及自动生成的备份目录。
    \item \textbf{src/}:存放软件的核心源代码及日志文件。
\end{itemize}

\section{软件配置指南}

在使用软件前,用户可根据自身需求调整 \texttt{config/config.yaml} 文件。该设计使得软件具有极高的灵活性,能够适应不同格式的 Excel 表格。

\subsection{Excel 列映射配置}
软件允许用户自定义单词、音标、释义等信息在 Excel 中的列位置。默认配置如下:

%-----------------------------------------------------------
% 图片位置 2:配置文件代码截图
% 请截取 config.yaml 文件的内容
%-----------------------------------------------------------
\begin{figure}[H]
    \centering
    \includegraphics[width=\textwidth]{config_code.png}
    \caption{配置文件参数详情}
    \label{fig:config}
\end{figure}

\begin{itemize}
    \item \textbf{file\_path}:指定待处理的 Excel 文件相对路径。
    \item \textbf{columns}:定义各数据项对应的列号(A列对应1,B列对应2,以此类推)。
    \begin{itemize}
        \item \texttt{word: 1} (A列存放英文单词)
        \item \texttt{phonetic: 2} (B列用于回填音标)
        \item \texttt{definition: 3} (C列用于回填释义)
        \item \texttt{example: 4} (D列用于回填例句)
        \item \texttt{status: 6} (F列用于记录处理状态,避免重复抓取)
    \end{itemize}
\end{itemize}

\subsection{网络请求配置}
为了应对网络波动及防止被目标服务器封禁,软件内置了智能请求控制策略:
\begin{itemize}
    \item \textbf{delay}:请求间隔时间(默认0.3秒),适当的延时可以模拟人类操作。
    \item \textbf{retries}:失败重试次数(默认3次),当网络超时,系统会自动重试。
    \item \textbf{user\_agent}:浏览器伪装标识,确保请求的合法性。
\end{itemize}

\section{操作流程详解}

\subsection{第一步:准备数据源}
打开 \texttt{xlsx} 目录下的 \texttt{words.xlsx} 文件。在第一列(或配置文件中指定的 Word 列)纵向输入需要学习的英文单词。其他列留空即可,软件会自动处理。

\textbf{注意}:数据准备完成后,请务必\textbf{关闭 Excel 文件},否则软件无法获得写入权限。

%-----------------------------------------------------------
% 图片位置 3:原始Excel文件
% 截取只有英文单词的Excel界面
%-----------------------------------------------------------
\begin{figure}[H]
    \centering
    \includegraphics[width=0.8\textwidth]{excel_raw.png}
    \caption{数据源准备:仅需输入英文单词}
    \label{fig:raw}
\end{figure}

\subsection{第二步:启动程序}
双击运行 \texttt{run.bat} 或在命令行中执行 \texttt{python src/main.py}。程序启动后,系统将按照下图所示的逻辑进行处理:

%-----------------------------------------------------------
% 修复版逻辑流程图
% 请确保导言区引用了 \usetikzlibrary{positioning}
% (mancls.cls 默认可能没有包含 positioning,我在下面代码里强制加上样式定义)
%-----------------------------------------------------------

\begin{figure}[H]
\centering
\begin{tikzpicture}[
    node distance=1.2cm, % 调整基础间距
    auto,
    % 重新定义样式,确保宽度足够,文字不换行
    startstop/.style={rectangle, rounded corners, minimum width=3.5cm, minimum height=1cm, text centered, draw=black, fill=red!30},
    io/.style={trapezium, trapezium left angle=70, trapezium right angle=110, minimum width=3.5cm, minimum height=1cm, text centered, draw=black, fill=blue!30},
    process/.style={rectangle, minimum width=3.5cm, minimum height=1cm, text centered, draw=black, fill=orange!30},
    decision/.style={diamond, minimum width=3cm, minimum height=1cm, text centered, draw=black, fill=cyan!30, inner sep=0pt},
    arrow/.style={thick,->,>=stealth}
]

    % 1. 放置节点 (使用 below of 可能会重叠,这里通过手动调节 node distance 解决)
    \node (start) [startstop] {开始运行};
    \node (config) [io, below of=start, node distance=1.8cm] {读取 config.yaml};
    \node (backup) [process, below of=config, node distance=1.8cm] {执行自动备份};
    
    % 判定框通常比较大,给它更多空间
    \node (loop) [decision, below of=backup, node distance=2.2cm] {遍历行数据}; 
    
    % 检查框
    \node (check) [decision, below of=loop, node distance=2.5cm] {检查状态};
    
    % 右侧的处理框
    \node (fetch) [process, right of=check, node distance=4.5cm] {网络抓取数据};
    
    % 写入框
    \node (save) [process, below of=fetch, node distance=1.8cm] {写入Excel单元格};
    
    % 结束框 (放在左下角)
    \node (end) [startstop, left of=check, node distance=4cm] {保存并结束};

    % 2. 绘制连接线
    \draw [arrow] (start) -- (config);
    \draw [arrow] (config) -- (backup);
    \draw [arrow] (backup) -- (loop);
    
    % 遍历逻辑
    \draw [arrow] (loop) -- node[anchor=east] {有数据} (check);
    \draw [arrow] (loop) -| node[anchor=south] {无数据} (end);
    
    % 检查逻辑
    \draw [arrow] (check) -- node[anchor=south] {未处理} (fetch);
    
    % 这里的线条容易重叠,我们让它绕一下
    % 如果“已处理”,直接回到循环
    \draw [arrow] (check.west) -- node[anchor=south] {跳过} ++(-1.5,0) |- (loop.west);
    
    \draw [arrow] (fetch) -- (save);
    
    % 写入后回到循环
    \draw [arrow] (save.south) |- (0,-9.5) -- (-5,-9.5) |- (loop.west);

\end{tikzpicture}
\caption{软件核心处理逻辑流程图}
\label{fig:flowchart}
\end{figure}

\subsection{第三步:监控运行进度}
程序运行过程中,控制台会实时输出当前的进度日志。
\begin{itemize}
    \item \textbf{INFO: 已处理 [单词]}:表示该单词获取成功。
    \item \textbf{WARNING: 部分成功}:表示可能缺失例句或音标,但核心释义已获取。
    \item \textbf{ERROR: 处理失败}:表示单词拼写错误或词典中未收录。
\end{itemize}

%-----------------------------------------------------------
% 图片位置 4:控制台运行截图
%-----------------------------------------------------------
\begin{figure}[H]
    \centering
    \includegraphics[width=0.8\textwidth]{run_console.png}
    \caption{控制台实时显示处理进度}
    \label{fig:console}
\end{figure}

\subsection{第四步:自动备份机制}
为了防止意外情况导致用户辛辛苦苦收集的单词丢失,软件在每次运行的瞬间,都会在 \texttt{xlsx/backup} 目录下生成一个以当前时间命名的副本文件。

%-----------------------------------------------------------
% 图片位置 5:备份文件夹截图
%-----------------------------------------------------------
\begin{figure}[H]
    \centering
    \includegraphics[width=0.6\textwidth]{backup_folder.png}
    \caption{系统自动生成的历史备份文件}
    \label{fig:backup}
\end{figure}

\subsection{第五步:查看最终结果}
待控制台显示“处理完成”后,打开 \texttt{words.xlsx}。此时,表格中的音标、释义、例句列均已自动填充完毕,且所有单元格均已应用了统一的字体和自动换行格式。

%-----------------------------------------------------------
% 图片位置 6:处理结果截图
%-----------------------------------------------------------
\begin{figure}[H]
    \centering
    \includegraphics[width=0.9\textwidth]{excel_result.png}
    \caption{自动填充完毕的词汇学习表}
    \label{fig:result}
\end{figure}

\section{特色功能演示}

\subsection{无损格式化技术}
本系统的一大特色是“无损格式化”。在调整字体(如统一为 Arial 12号)和对齐方式时,系统会通过深拷贝技术,完整保留用户手动设置的单元格背景色(如标记为红色的生词)和字体颜色。

%-----------------------------------------------------------
% 图片位置 7:格式细节截图
%-----------------------------------------------------------
\begin{figure}[H]
    \centering
    \includegraphics[width=0.8\textwidth]{format_detail.png}
    \caption{格式化后依然保留了用户的高亮标记}
    \label{fig:format}
\end{figure}

\subsection{失败重试与日志记录}
对于因拼写错误或网络原因导致获取失败的单词,系统会自动将其记录在 \texttt{failed\_words.txt} 文件中,并在 Excel 的“错误历史”列进行标记,方便用户后续进行人工核对。

\section{常见问题与故障排除}

\begin{table}[h]
\centering
\caption{常见问题解决方案表}
\begin{tabular}{|p{4cm}|p{5cm}|p{5cm}|}
\hline
\textbf{问题现象} & \textbf{可能原因} & \textbf{解决方案} \\
\hline
Permission denied 报错 & Excel 文件正在被其他程序占用 & 请关闭所有打开的 Excel 窗口后重试。 \\
\hline
Connection Error & 网络连接中断或不稳定 & 检查网络设置,增加 config.yaml 中的 retries 次数。 \\
\hline
所有单词均失败 & IP 地址可能被目标服务器暂时限制 & 增加 config.yaml 中的 delay 值(如设为 1.0),或稍后重试。 \\
\hline
中文乱码 & 配置文件编码格式错误 & 确保 config.yaml 保存为 UTF-8 编码格式。 \\
\hline
\end{tabular}
\end{table}

\end{document}